% interactcadsample.tex
% v1.03 - April 2017

\documentclass[]{interact}

\usepackage{epstopdf}% To incorporate .eps illustrations using PDFLaTeX, etc.
\usepackage{subfigure}% Support for small, `sub' figures and tables
%\usepackage[nolists,tablesfirst]{endfloat}% To `separate' figures and tables from text if required

\usepackage{natbib}% Citation support using natbib.sty
\bibpunct[, ]{(}{)}{;}{a}{}{,}% Citation support using natbib.sty
\renewcommand\bibfont{\fontsize{10}{12}\selectfont}% Bibliography support using natbib.sty

\theoremstyle{plain}% Theorem-like structures provided by amsthm.sty
\newtheorem{theorem}{Theorem}[section]
\newtheorem{lemma}[theorem]{Lemma}
\newtheorem{corollary}[theorem]{Corollary}
\newtheorem{proposition}[theorem]{Proposition}

\theoremstyle{definition}
\newtheorem{definition}[theorem]{Definition}
\newtheorem{example}[theorem]{Example}

\theoremstyle{remark}
\newtheorem{remark}{Remark}
\newtheorem{notation}{Notation}

% Pandoc syntax highlighting
\usepackage{color}
\usepackage{fancyvrb}
\newcommand{\VerbBar}{|}
\newcommand{\VERB}{\Verb[commandchars=\\\{\}]}
\DefineVerbatimEnvironment{Highlighting}{Verbatim}{commandchars=\\\{\}}
% Add ',fontsize=\small' for more characters per line
\usepackage{framed}
\definecolor{shadecolor}{RGB}{248,248,248}
\newenvironment{Shaded}{\begin{snugshade}}{\end{snugshade}}
\newcommand{\AlertTok}[1]{\textcolor[rgb]{0.94,0.16,0.16}{#1}}
\newcommand{\AnnotationTok}[1]{\textcolor[rgb]{0.56,0.35,0.01}{\textbf{\textit{#1}}}}
\newcommand{\AttributeTok}[1]{\textcolor[rgb]{0.13,0.29,0.53}{#1}}
\newcommand{\BaseNTok}[1]{\textcolor[rgb]{0.00,0.00,0.81}{#1}}
\newcommand{\BuiltInTok}[1]{#1}
\newcommand{\CharTok}[1]{\textcolor[rgb]{0.31,0.60,0.02}{#1}}
\newcommand{\CommentTok}[1]{\textcolor[rgb]{0.56,0.35,0.01}{\textit{#1}}}
\newcommand{\CommentVarTok}[1]{\textcolor[rgb]{0.56,0.35,0.01}{\textbf{\textit{#1}}}}
\newcommand{\ConstantTok}[1]{\textcolor[rgb]{0.56,0.35,0.01}{#1}}
\newcommand{\ControlFlowTok}[1]{\textcolor[rgb]{0.13,0.29,0.53}{\textbf{#1}}}
\newcommand{\DataTypeTok}[1]{\textcolor[rgb]{0.13,0.29,0.53}{#1}}
\newcommand{\DecValTok}[1]{\textcolor[rgb]{0.00,0.00,0.81}{#1}}
\newcommand{\DocumentationTok}[1]{\textcolor[rgb]{0.56,0.35,0.01}{\textbf{\textit{#1}}}}
\newcommand{\ErrorTok}[1]{\textcolor[rgb]{0.64,0.00,0.00}{\textbf{#1}}}
\newcommand{\ExtensionTok}[1]{#1}
\newcommand{\FloatTok}[1]{\textcolor[rgb]{0.00,0.00,0.81}{#1}}
\newcommand{\FunctionTok}[1]{\textcolor[rgb]{0.13,0.29,0.53}{\textbf{#1}}}
\newcommand{\ImportTok}[1]{#1}
\newcommand{\InformationTok}[1]{\textcolor[rgb]{0.56,0.35,0.01}{\textbf{\textit{#1}}}}
\newcommand{\KeywordTok}[1]{\textcolor[rgb]{0.13,0.29,0.53}{\textbf{#1}}}
\newcommand{\NormalTok}[1]{#1}
\newcommand{\OperatorTok}[1]{\textcolor[rgb]{0.81,0.36,0.00}{\textbf{#1}}}
\newcommand{\OtherTok}[1]{\textcolor[rgb]{0.56,0.35,0.01}{#1}}
\newcommand{\PreprocessorTok}[1]{\textcolor[rgb]{0.56,0.35,0.01}{\textit{#1}}}
\newcommand{\RegionMarkerTok}[1]{#1}
\newcommand{\SpecialCharTok}[1]{\textcolor[rgb]{0.81,0.36,0.00}{\textbf{#1}}}
\newcommand{\SpecialStringTok}[1]{\textcolor[rgb]{0.31,0.60,0.02}{#1}}
\newcommand{\StringTok}[1]{\textcolor[rgb]{0.31,0.60,0.02}{#1}}
\newcommand{\VariableTok}[1]{\textcolor[rgb]{0.00,0.00,0.00}{#1}}
\newcommand{\VerbatimStringTok}[1]{\textcolor[rgb]{0.31,0.60,0.02}{#1}}
\newcommand{\WarningTok}[1]{\textcolor[rgb]{0.56,0.35,0.01}{\textbf{\textit{#1}}}}

% tightlist command for lists without linebreak
\providecommand{\tightlist}{%
  \setlength{\itemsep}{0pt}\setlength{\parskip}{0pt}}



\usepackage{lscape}
\usepackage{hyperref}
\usepackage[utf8]{inputenc}
\def\tightlist{}
\usepackage{setspace}
\doublespacing

\usepackage{booktabs}
\usepackage{longtable}
\usepackage{array}
\usepackage{multirow}
\usepackage{wrapfig}
\usepackage{float}
\usepackage{colortbl}
\usepackage{pdflscape}
\usepackage{tabu}
\usepackage{threeparttable}
\usepackage{threeparttablex}
\usepackage[normalem]{ulem}
\usepackage{makecell}
\usepackage{xcolor}

\begin{document}


\articletype{}

\title{Automated assessment of residual plots with computer vision
models}


\author{\name{Weihao Li$^{a}$, Dianne Cook$^{a}$, Emi Tanaka$^{b,
c}$, Susan VanderPlas$^{d}$, Klaus Ackermann$^{a}$}
\affil{$^{a}$Department of Econometrics and Business Statistics, Monash
University, Clayton, VIC, Australia; $^{b}$Biological Data Science
Institute, Australian National University, Acton, ACT,
Australia; $^{c}$Research School of Finance, Actuarial Studies and
Statistics, Australian National University, Acton, ACT,
Australia; $^{d}$Department of Statistics, University of Nebraska,
Lincoln, Nebraska, USA}
}

\thanks{CONTACT Weihao
Li. Email: \href{mailto:weihao.li@monash.edu}{\nolinkurl{weihao.li@monash.edu}}, Dianne
Cook. Email: \href{mailto:dicook@monash.edu}{\nolinkurl{dicook@monash.edu}}, Emi
Tanaka. Email: \href{mailto:emi.tanaka@anu.edu.au}{\nolinkurl{emi.tanaka@anu.edu.au}}, Susan
VanderPlas. Email: \href{mailto:susan.vanderplas@unl.edu}{\nolinkurl{susan.vanderplas@unl.edu}}, Klaus
Ackermann. Email: \href{mailto:Klaus.Ackermann@monash.edu}{\nolinkurl{Klaus.Ackermann@monash.edu}}}

\maketitle

\begin{abstract}
Plotting residuals is a standard practice in linear regression
diagnostics, essential for identifying deviations from model assumptions
such as linearity, homoscedasticity, and normality. Visual inference
provides an inferential framework to assess whether residual plots
contain patterns inconsistent with model assumptions, typically using a
lineup protocol. However, the lineup protocol's reliance on human
judgment limits its scalability. This study addresses this limitation by
automating the interpretation of residual plots using computer vision
models. We develop a distance measure based on Kullback-Leibler
divergence to quantify the disparity between the residual distribution
of a fitted classical normal linear regression model and the reference
distribution. We propose a computer vision model to estimate this
distance from residual plots, facilitating formal statistical testing
and bootstrapping techniques to assess model specification. Our computer
vision model shows strong performance, though it performs slightly less
effectively on non-linearity visual patterns. The statistical tests
based on the estimated distance exhibit lower sensitivity than
conventional tests but higher sensitivity than human visual tests.
Examples demonstrate the method's effectiveness across different
scenarios, highlighting its value in automating the diagnostic process
and supplementing traditional methods.
\end{abstract}

\begin{keywords}
TBD
\end{keywords}

\section{Introduction}\label{introduction}

Plotting residuals is commonly regarded as a standard practice in linear
regression diagnostics
\citep[see][]{cook1982residuals, belsley1980regression}. This visual
assessment plays a crucial role in identifying deviations from model
assumptions, such as linearity, homoscedasticity, and normality. It also
helps in understanding the goodness of fit and various characteristics
of the model.

Generating a residual plot in most statistical software is often as
straightforward as executing a line of code or clicking a button.
However, accurately interpreting a residual plot can be challenging. A
residual plot can exhibit various visual features, but it is crucial to
recognize that some may arise from the characteristics of predictors and
the inherent randomness of the error, rather than indicating a violation
of model assumptions \citep{li2023plot}. Consider Figure
\ref{fig:false-finding} as an example, the residuals display a
triangular shape pointing to the left. While this might suggest
heteroskedasticity, it is important to avoid over-interpreting this
visual pattern. In this case, the fitted regression model is correctly
specified, and the triangular shape is actually a result of the skewed
distribution of the predictors, rather than indicating a flaw in the
model.

The concept of visual inference, as proposed by
\citet{buja2009statistical}, provides an inferential framework to assess
whether residual plots indeed contain visual patterns inconsistent with
the model assumptions. The fundamental idea involves testing whether the
true residual plot visually differs significantly from null plots, where
null plots are plotted with residuals generated from the residual
rotation distribution \citep{langsrud2005rotation}, which is a
distribution consistent with the null hypothesis \(H_0\) that the linear
regression model is correctly specified. Typically, the visual test is
accomplished through the lineup protocol, where the real residual plot
is embedded within a lineup alongside several null plots. If the real
residual plot can be distinguished from the lineup, it provides evidence
for rejecting \(H_0\).

The practice of delivering a residual plot as a lineup is generally
regarded as a valuable approach. Beyond its application in residual
diagnostics, the lineup protocol has integrated into the analysis of
diverse subjects. For instance,
\cite{loy2013diagnostic, loy2014hlmdiag, loy2015you} illustrated its
applicability in diagnosing hierarchical linear models. Additionally,
\citet{widen2016graphical} demonstrated its utility in geographical
research, while \citet{krishnan2021hierarchical} explored its
effectiveness in forensic examinations.

However, as pointed out by \citet{li2023plot}, a primary limitation of
the lineup protocol lies in its reliance on human judgments. Unlike
conventional statistical tests that can be performed computationally in
statistical software, the lineup protocol requires human evaluation of
images. This characteristic makes it less suitable for large-scale
applications, given the associated high labour costs and time
requirements.

There is a substantial need to develop an approach that alleviates
analysts' workload by automating repetitive tasks and providing
standardized results in a controlled environment. The large-scale
evaluation of lineups is impractical without the use of technology and
machines.

The utilization of computers to interpret data plots has a rich history,
with early efforts such as ``Scagnostics'' by \citet{tukey1985computer},
focusing on scatter plot diagnostics. \citet{wilkinson2005graph}
expanded on this work, introducing graph theoretic scagnostics, which
encompassed computable measures applied to planar proximity graphs.
These measures, including, but not limited to, ``Outlying'', ``Skinny'',
``Stringy'', ``Straight'', ``Monotonic'', ``Skewed'', ``Clumpy'', and
``Striated'' aimed to characterize outliers, shape, density, trend,
coherence and other characteristics of the data. While this approach has
been inspiring, there is a recognition \citep{buja2009statistical} that
it may not capture all the necessary visual features that differentiate
true residual plots from null plots. A more promising alternative
entails enabling machines to learn the function for extracting visual
features from residual plots. Essentially, this means empowering
computers to discern the crucial visual features for residual
diagnostics and determining the method to extract them.

Modern computer vision models are well-suited for addressing this
challenge. They rely on deep neural networks with convolutional layers
\citep{fukushima1982neocognitron}. These layers leverage hierarchical
patterns in data, downsizing and transforming images by summarizing
information in a small space. Numerous studies have demonstrated the
efficacy of convolutional layers in addressing various vision tasks,
including image recognition \citep{rawat2017deep}. Despite the
widespread use of computer vision models in fields like computer-aided
diagnosis \citep{lee2015image}, pedestrian detection
\citep{brunetti2018computer}, and facial recognition
\citep{emami2012facial}, their application in reading data plots remains
limited. While some studies have explored the use of computer vision
models for tasks such as reading recurrence plots for time series
regression \citep{ojeda2020multivariate}, time series classification
\citep{chu2019automatic, hailesilassie2019financial, hatami2018classification, zhang2020encoding},
anomaly detection \citep{chen2020convolutional}, and pairwise causality
analysis \citep{singh2017deep}, the application of reading residual
plots with computer vision models represents a relatively new field of
study.

In this paper, we develop computer vision models and integrate them into
the residual plots diagnostics workflow, addressing the need for
automated visual inference. The paper is structured as follows. Section
\ref{model-specifications} discusses various specifications of the
computer vision models. Section \ref{distance-between-residual-plots}
defines the distance measure used to measure model violations. Section
\ref{distance-estimation} explains how the computer vision models
estimate this distance measure. Section \ref{statistical-testing} covers
the statistical testing based on the estimated distance. Sections
\ref{data-generation}, \ref{model-architecture}, and
\ref{model-training} detail the data preparation, model architecture,
and training process, respectively. The results are presented in Section
\ref{results}. Finally, we conclude with a discussion of our findings
and propose ideas for future research directions.

\begin{figure}[!h]

{\centering \includegraphics[width=1\linewidth]{paper_files/figure-latex/false-finding-1} 

}

\caption{An example residual vs fitted values plot (red line indicates 0). The vertical spread of the data points varies with the fitted values. This often indicates the existence of heteroskedasticity. The Breusch-Pagan test rejects this residual plot at 95\% significance level ($p\text{-value} = 0.046$).}\label{fig:false-finding}
\end{figure}

\section{Model Specifications}\label{model-specifications}

There are various specifications of the computer vision model that can
be used to assess residual plots. We discuss these specifications below
focusing on two key components of the model formula: the input and the
output format.

\subsection{Input Formats}\label{input-formats}

Deep learning models are in general very sensitive to the input data.
The quality and relevance of the input data greatly influence the
model's capacity to generate insightful and meaningful results. There
are several designs of the input format can be considered.

A straightforward design involves feeding a vector of residuals along
with a vector of fitted values, essentially providing all the necessary
information for creating a residuals vs fitted values plot. However, a
drawback of this method is the dynamic input size, which changes based
on the number of observations. For modern computer vision models
implemented in mainstream software like TensorFlow
\citep{abadi2016tensorflow}, the input shape is typically fixed. One
solution is to pad the input vectors with leading or trailing zeros when
the input tensor expects longer vectors, but it may fail if the input
vector surpasses the designed length. Another strategy is to summarize
the residuals and fitted values separately using histograms and utilize
the counts as the input. By controlling the number of bins in the
histograms, it becomes possible to provide fixed-length input vectors.
Still, since histograms only capture the marginal distribution of
residuals and fitted values respectively, they can not be used to
differentiate visual patterns with same marginal distributions but
different joint distributions.

Another design involves using an image as input. The primary advantage
of this design, as opposed to the vector format, is the availability of
the existing and sophisticated image processing architectures developed
over the years, such as the VGG16 architecture proposed in
\citet{simonyan2014very}. These architectures can effectively capture
and summarize spatial information from nearby pixels, which is less
straightforward with vector input. The main considerations are the image
resolution and the aesthetics of the residual plot. In general, higher
resolution provides more information to the model but comes with the
trade-off of increased complexity and greater difficulty in training. As
for the aesthetics of the residual plot, a practical solution is to
consistently present residual plots in the same style to the model. This
implies that the model can not accept arbitrary images as input but
requires the use of the same pre-processing pipeline to convert
residuals and fitted values into a standardized-style residual plot.

Providing multiple residual plots to the model, such as a pair of plots,
a triplet or a lineup is also a possible option.
\citet{chopra2005learning} have shown that computer vision models
designed for image comparison can assess whether a pair of images are
similar or dissimilar. Applied to our specific problem, we can define
null plots of a fitted regression model to be similar to each other,
while considering true residual plots to be distinct from null plots of
any fitted regression model. A triplet constitutes a set of three
images, denoted as \(image_1\), \(image_2\) and \(image_3\). It is often
used to predict whether \(image_2\) or \(image_3\) is more similar to
\(image_1\), proving particularly useful for establishing rankings
between samples. For this setup, we can apply the same criteria to
define similarity between images. However, it is important to note that
these two approaches usually require additional considerations regarding
the loss function and, at times, non-standard training processes due to
shared weights between different convolutional blocks.

Presenting a lineup to a model aligns closely with the lineup protocol.
However, as the number of residual plots in a lineup increases, the
resolution of the input image grows rapidly, posing challenges in
training the model. We experimented with this approach in a pilot study,
but the performance of the trained model was sub-optimal.

We did not explore all the mentioned input formats due to the
significant costs tied to data preparation and model training. Taking
into account the implementation cost and the need for model
interpretability, we settled on the single residual plot input format.

\subsection{Output Formats}\label{output-formats}

Given that the input is a single residual plot represented as a
fixed-resolution image, the output from the computer vision model can
take one of two forms: binary or numeric. This choice determines whether
the model belongs to a classification model or a regression model. The
binary outcome encoded as \(0\) and \(1\) could be used to represent
whether the input image is a null plot, or whether the input image would
be rejected in a visual test conducted by humans. Training a model
following the latter option requires data from prior human subject
experiments, presenting difficulties in controlling the quality of data
due to variations in experimental settings across different studies.
Additionally, some visual inference experiments are unrelated to linear
regression models or residual plot diagnostics, resulting in a limited
amount of available training data.

Alternatively, the output could be a meaningful and interpretable
numerical measure useful for assessing residual plots, such as the
strength of suspicious visual patterns reflecting the extent of model
violations, or the difficulty index for identifying whether a residual
plot has no issues. However, these numeric measures are often informally
used in daily communication but are not typically formalized or
rigorously defined. For the purpose of training a model, this numeric
measure has to be quantifiable.

In this study, we chose to define and use a distance measure to quantify
the difference between the residual plot and the null plots.
\citet{vo2016localizing} have also demonstrated that defining a proper
distance between images can enhance the matching accuracy in image
search compared to a binary outcome model.

\section{Distance Between Residual
Plots}\label{distance-between-residual-plots}

In a visual test, the observer will be asked to choose one or more plots
that stand out as most distinct from others in a given lineup. To
develop a computer vision model for assessing residual plots within the
visual inference framework, it is important to precisely define a
numerical measure of ``difference'' or ``distance'' between plots. This
distance can take the form of a basic statistical operation on pixels,
such as the sum of square differences. Alternatively, it could involve
established image similarity metrics like the Structural Similarity
Index Measure \citep{wang2004image}. The challenge lies in the fact that
metrics tailored for image comparison may not be suitable for evaluating
data plots, where only essential plot elements require assessment
\citep{chowdhury2018measuring}. Furthermore, scagnostics mentioned in
Section \ref{introduction} could be used to construct distance measures
for residual plots comparison, but the functional form still needs to be
carefully refined to accurately reflect the extent of the violations.

\subsection{Residual Distribution}\label{residual-distribution}

The distance measure proposed in this study takes into account the fact
that we tried to measure how different a residual plot is from a good
residual plot, or in other words, how different a given fitted
regression model is from a correctly specified model.

For a classical normal linear regression model, residuals
\(\boldsymbol{e}\) are derived from the fitted values
\(\hat{\boldsymbol{y}}\) and observed values \(\boldsymbol{y}\). Suppose
the data generating process is known and the regression model is
correctly specified, by the Frisch-Waugh-Lowell theorem
\citep{frisch1933partial}, residuals \(\boldsymbol{e}\) can also be
treated as random variables and written as a linear transformation of
the error \(\boldsymbol{\varepsilon}\) formulated as
\(\boldsymbol{e} = \boldsymbol{R}\boldsymbol{\varepsilon}\), where
\(\boldsymbol{R}=\boldsymbol{I}_n -\boldsymbol{X}(\boldsymbol{X}'\boldsymbol{X})^{-1}\boldsymbol{X}'\)
is the residual operator and an idempotent matrix, \(\boldsymbol{X}\) is
the design matrix, \(\boldsymbol{I}_n\) is a \(n\) by \(n\) identity
matrix, and \(n\) is the number of observations.

One of the assumptions of the classical normal linear regression model
is the error \(\boldsymbol{\varepsilon}\) follows a multivariate normal
distribution with zero mean and constant variance, i.e.,
\(\boldsymbol{\varepsilon} \sim N(\boldsymbol{0}_n,\sigma^2\boldsymbol{I}_n)\).
It can be known that residuals \(\boldsymbol{e}\) also follow a certain
probability distribution transformed from the multivariate normal
distribution, which will be denoted as \(Q\). This reference
distribution \(Q\) summarizes what good residuals should follow given
the design matrix \(\boldsymbol{X}\) is known and fixed.

Suppose the design matrix \(\boldsymbol{X}\) has linearly independent
columns, the trace of the hat matrix will equal to the number of columns
in \(\boldsymbol{X}\) denoted as \(k\). As a result, the rank of
\(\boldsymbol{R}\) is \(n - k\), and \(Q\) is a degenerate multivariate
distribution. To capture the characteristics of \(Q\), such as moments,
we can simulate a large numbers of \(\boldsymbol{\varepsilon}\) and
transform it to \(\boldsymbol{e}\) to get the empirical estimates. For
simplicity, in this study, we replaced the variance-covariance matrix of
residuals
\(\text{cov}(\boldsymbol{e}, \boldsymbol{e}) = \boldsymbol{R}\sigma^2\boldsymbol{R}' = \boldsymbol{R}\sigma^2\)
with a full-rank diagonal matrix
\(\text{diag}(\boldsymbol{R}\sigma^2)\), where \(\text{diag}(.)\) sets
the non-diagonal entries of a matrix to zeros. The resulting
distribution for \(\boldsymbol{Q}\) is
\(N(\boldsymbol{0}_n, \text{diag}(\boldsymbol{R}\sigma^2))\).

Distribution \(Q\) is derived from the correctly specified model.
However, if the model is misspecified, then the actual distribution of
residuals denoted as \(P\), will be different from \(Q\). For example,
if the data generating process contains variables correlated with any
column of \(\boldsymbol{X}\) but missing from \(\boldsymbol{X}\),
causing an omitted variable problem, \(P\) will be different from \(Q\)
because the residual operator obtained from the fitted regression model
will not be the same as \(\boldsymbol{R}\). Besides, if the
\(\boldsymbol{\varepsilon}\) follows a non-normal distribution such as a
multivariate lognormal distribution, \(P\) will usually be skewed and
has a long tail.

\subsection{\texorpdfstring{Distance of \(P\) from
\(Q\)}{Distance of P from Q}}\label{distance-of-p-from-q}

Define a proper distance between distributions is usually easier than
define a proper distance between data plots. Given the true residual
distribution \(Q\) and the reference residual distribution \(P\), we
used a distance measure based on Kullback-Leibler divergence
\citep{kullback1951information} to quantify the difference between two
distributions

\begin{align}
\label{eq:kl-0}
D &= \log\left(1 + D_{KL}\right), \\
\label{eq:kl-1}
D_{KL} &= \int_{\mathbb{R}^{n}}\log\frac{p(\boldsymbol{e})}{q(\boldsymbol{e})}p(\boldsymbol{e})d\boldsymbol{e},
\end{align}

\noindent where \(p(.)\) and \(q(.)\) are the probability density
functions for distribution \(P\) and distribution \(Q\) respectively.

This distance measure was first proposed in \citet{li2023plot}. It was
mainly designed for measuring the effect size of non-linearity and
heteroskedasticity in a residual plot. \citet{li2023plot} have derived
that, for a classical normal linear regression model that omits a
necessary higher-order predictors \(\boldsymbol{Z}\), and incorrectly
assumes
\(\boldsymbol{\varepsilon} \sim N(\boldsymbol{0}_n,\sigma^2\boldsymbol{I}_n)\)
while in fact
\(\boldsymbol{\varepsilon} \sim N(\boldsymbol{0}_n, \boldsymbol{V})\)
with \(\boldsymbol{V}\) being an arbitrary symmetric positive
semi-definite matrix, \(Q\) can be represented as
\(N(\boldsymbol{R}\boldsymbol{Z}\boldsymbol{\beta}_z, \text{diag}(\boldsymbol{R}\boldsymbol{V}\boldsymbol{R}))\).
Note that the variance-covariance matrix is replaced with the diagonal
matrix to ensure it is a full-rank matrix.

Since both \(P\) and \(Q\) are adjusted to be multivariate normal
distributions, equation \ref{eq:kl-1} can be further expanded to

\begin{align}
\label{eq:kl-2}
D_{KL} &= \frac{1}{2}\left(\log\frac{|\text{diag}(\boldsymbol{W})|}{|\text{diag}(\boldsymbol{R}\sigma^2)|} - n + tr(\text{diag}(\boldsymbol{W})^{-1}\text{diag}(\boldsymbol{R}\sigma^2)) + \boldsymbol{\mu}_z'(\text{diag}(\boldsymbol{W}))^{-1}\boldsymbol{\mu}_z\right),
\end{align}

\noindent where
\(\boldsymbol{\mu}_z = \boldsymbol{R}\boldsymbol{Z}\boldsymbol{\beta}_z\),
and \(\boldsymbol{W} = \boldsymbol{R}\boldsymbol{V}\boldsymbol{R}\). The
assumed error variance \(\sigma^2\) is set to be
\(tr(\boldsymbol{V})/n\), which is the expectation of the estimated
variance.

\subsection{\texorpdfstring{Non-normal
\(P\)}{Non-normal P}}\label{non-normal-p}

For non-normal error \(\boldsymbol{\varepsilon}\), the true residual
distribution \(P\) is unlikely to be a multivariate normal distribution.
Thus, equation \ref{eq:kl-2} given in \citet{li2023plot} will not be
applicable to models violating the normality assumption.

To evaluate the Kullback-Leibler divergence of non-normal \(P\) from
\(Q\), the fallback is to solve equation \ref{eq:kl-1} numerically.
However, since \(\boldsymbol{e}\) is a linear transformation of
non-normal random variables, it is very common that the general form of
\(P\) is unknown, meaning that we can not easily compute
\(p(\boldsymbol{e})\) using a well-known probability density function.
Additionally, even if \(p(\boldsymbol{e})\) can be calculated for any
\(\boldsymbol{e} \in \mathbb{R}^n\), it will be very difficult to do
numerical integration over the \(n\) dimensional space, because \(n\)
could be potentially very large.

In order to approximate \(D_{KL}\) in a practically computable manner,
the elements of \(\boldsymbol{e}\) are assumed to be independent of each
other. This assumption solves both of the issues mentioned above. First,
we no longer need to integrate over \(n\) random variables. The result
of equation \ref{eq:kl-1} is now the sum of the Kullback-Leibler
divergence evaluated for each individual residual thanks for the
independence assumption. Second, it is not required to know the joint
probability density \(p(\boldsymbol{e})\) any more. Instead, the
evaluation of Kullback-Leibler divergence for an individual residual
relies on the knowledge of the marginal density \(p_i(e_i)\), where
\(e_i\) is the \(i\)-th residual for \(i = 1, ..., n\). This is much
easier to approximate through simulation. It is also worth mentioning
that this independence assumption generally will not hold, since
\(\text{cov}(e_i, e_j) \neq 0\) if \(\boldsymbol{R}_{ij} \neq 0\) for
any \(1 \leq i < j \leq n\), but its existence is essential for reducing
the computational cost.

Given \(\boldsymbol{X}\) and \(\boldsymbol{\beta}\), the algorithm for
approximating equation \ref{eq:kl-1} starts from simulating \(m\) sets
of observed values \(\boldsymbol{y}\) according to the data generating
process. The observed values are stored in a matrix \(\boldsymbol{A}\)
with \(n\) rows and \(m\) columns, where each column of
\(\boldsymbol{A}\) is a set of observed values. Then, we can get \(m\)
sets of realized values of \(\boldsymbol{e}\) stored in the matrix
\(\boldsymbol{B}\) by applying the residual operator
\(\boldsymbol{B} = \boldsymbol{R}\boldsymbol{A}\). Furthermore, kernel
density estimation (KDE) with Gaussian kernel and optimal bandwidth
selected by the Silverman's rule of thumb \citep{silverman2018density}
is applied on each row of \(B\) to estimate \(p_i(e_i)\) for
\(i = 1, ..., n\). The KDE computation can be done by the
\texttt{density} function in R.

Since the Kullback-Leibler divergence can be viewed as the expectation
of the log-likelihood ratio between distribution \(P\) and distribution
\(Q\) evaluated on distribution \(P\), we can reuse the simulated
residuals in matrix \(\boldsymbol{B}\) to estimate the expectation by
the sample mean. With the independence assumption, for non-normal \(P\),
\(D_{KL}\) can be approximated by

\begin{align}
\label{eq:kl-3}
D_{KL} &\approx \sum_{i = 1}^{n} \hat{D}_{KL}^{(i)}, \\
\hat{D}_{KL}^{(i)} &= \frac{1}{m}\sum_{j = 1}^{m} log\frac{\hat{p_i}(B_{ij})}{q(B_{ij})},
\end{align}

\noindent where \(\hat{D}_{KL}^{(i)}\) is the estimator of the
Kullback-Leibler divergence for an individual residual \(e_i\),
\(B_{ij}\) is the \(i\)-th row and \(j\)-th column entry of the matrix
\(B\), \(\hat{p_i}(.)\) is the kernel density estimator of \(p_i(.)\),
\(q(.)\) is the normal density function with mean zero and an assumed
variance estimated as
\(\widehat{\sigma^2} = \sum_{b \in vec(B)}(b - \sum_{b \in vec(B)} b/nm)^2/(nm - 1)\),
and \(vec(.)\) is the vectorization operator which turns a
\(n \times m\) matrix into a \(nm \times 1\) column vector by stacking
the columns of the matrix on top of each other.

\section{Distance Estimation}\label{distance-estimation}

In the previous sections, we have defined a distance measure given in
equation \ref{eq:kl-0} for quantifying the difference between the true
residual distribution \(P\) and an ideal reference distribution \(Q\).
It can be noticed that this distance measure can only be computed when
the data generating process is known. In reality, we often have no
knowledge about the data generating process, otherwise we do not need to
fit a linear regression model in the first place.

We tried to train a computer vision model to estimate this distance
measure with a residual plot. Let \(D\) be the result of equation
\ref{eq:kl-0}, and our estimator \(\hat{D}\) is formulated as

\begin{equation}
\label{eq:d-approx}
\hat{D} = f_{CV}(V_{h \times w}(\boldsymbol{e}, \hat{\boldsymbol{y}})),
\end{equation}

\noindent where \(V_{h \times w}(.)\) is a plotting function that saves
a residuals vs fitted values plot with fixed aesthetic as an RGB image
with \(h \times w\) pixels, \(f_{CV}(.)\) is a computer vision model
which takes an \(h \times w\) image as input and predicts the distance
in the domain \([0, +\infty)\).

With the estimated distance \(\hat{D}\), we will be able to know how
different the underlying distribution of the residuals is from a good
residual distribution. \(\hat{D}\) can also be used as an index of the
model violations indicating the strength of the visual signal embedded
in the residual plot.

It is not expected that \(\hat{D}\) will be equal to original distance
\(D\). This is largely because information contained in a single
residual plot is limited and it may not be able to summarise all the
important characteristics of the residual distribution. For a given
residual distribution \(P\), many different residual plots can be
simulated, where many will share similar visual patterns, but some of
them could be visually very different from the rest, especially for
regression models with small \(n\). This suggests the error of the
estimation will vary depends on whether the input residual plot is
representative or not.

\section{Model Violations Index (MVI)}\label{sec:model-violations-index}

\(\hat{D}\) is an estimator of the difference between the true residual
distribution and the reference residual distribution. This difference
primarily arises from deviations in model assumptions. The magnitude of
\(D\) directly reflects the degree of these deviations, thus making
\(\hat{D}\) instrumental in forming a model violations index.

Note that Equation \ref{eq:kl-0} might be influenced by the number of
random variables involved in its evaluation. Generally, a larger number
of observations will lead to a greater distance \(D\). However, this
does not imply that \(\hat{D}\) fails to accurately represent the extent
of model violations. In fact, when examining residual plots with more
observations, we often observe a stronger visual signal strength, as the
underlying patterns are more likely to be revealed, except in cases of
significant overlapping.

For a consistent data generating process, \(D\) typically increases
logarithmically with the number of observations. This behaviour comes
from the relationship \(D = \text{log}(1 + D_{KL})\), where
\(D_{KL} = \sum_{i=1}^{n}D_{KL}^{(i)}\) under the assumption of
independence.

Therefore, the Model Violations Index (MVI) can be proposed as

\begin{equation}
\label{eq:mvi}
\text{MVI} = C + \hat{D} - log(n),
\end{equation}

\noindent where \(C\) is a large enough constant keeping the result
positive.

Figures \ref{fig:poly-index} and \ref{fig:heter-index} display the
residual plots for fitted models exhibiting varying degrees of
non-linearity and heteroskedasticity. Each residual plot's MVI is
computed using Equation \ref{eq:mvi} with \(C = 10\). When
\(\text{MVI} > 8\), the visual patterns are notably strong and easily
discernible by humans. In the range \(6 < \text{MVI} < 8\), the
visibility of the visual pattern diminishes as MVI decreases.
Conversely, when \(\text{MVI} < 6\), the visual pattern tends to become
relatively faint and challenging to observe. Table \ref{tab:mvi}
provides a summary of the MVI usage and it is applicable to other linear
regression models.

\begin{table}

\caption{\label{tab:mvi}Degree of model violations or the strength of the visual signals according to the Model Violation Index (MVI). The constant $C$ is set to be 10.}
\centering
\begin{tabular}[t]{ll}
\toprule
Degree of model violations & Range ($C$ = 10)\\
\midrule
Strong & $\text{MVI} > 8$\\
Moderate & $6 < \text{MVI} < 8$\\
Weak & $\text{MVI} < 6$\\
\bottomrule
\end{tabular}
\end{table}

\begin{figure}[!h]

{\centering \includegraphics[width=1\linewidth]{paper_files/figure-latex/poly-index-1} 

}

\caption{Residual plots generated from fitted models exhibiting varying degrees of non-linearity violations. The Model violations index (MVI) is displayed atop each residual plot. The non-linearity patterns are relatively strong for $MVI > 8$, and relatively weak for $MVI < 6$.}\label{fig:poly-index}
\end{figure}

\begin{figure}[!h]

{\centering \includegraphics[width=1\linewidth]{paper_files/figure-latex/heter-index-1} 

}

\caption{Residual plots generated from fitted models exhibiting varying degrees of heteroskedasticity violations. The Model violations index (MVI) is displayed atop each residual plot. The heteroskedasticity patterns are relatively strong for $MVI > 8$, and relatively weak for $MVI < 6$.}\label{fig:heter-index}
\end{figure}

\section{Statistical testing}\label{statistical-testing}

\subsection{Lineup Evaluation}\label{lineup-evaluation}

Theoretically, the distance \(D\) for a correctly specified model is
\(0\), because \(P\) will be the same as \(Q\). However, the computer
vision model may not necessary predict \(0\) for a null plot. Using
Figure \ref{fig:false-finding} as an example, it contains a visual
pattern which is an indication of heteroskedasticity. We would not
expect the model to be able to magically tell if the suspicious pattern
is caused by the skewed distribution of the fitted values or the
existence of heteroskedasticity. Additionally, some null plots could
have outliers or strong visual patterns due to randomness, and a
reasonable model will try to summarise those information into the
prediction, resulting in \(\hat{D} > 0\).

This property is not an issue if \(\hat{D} \gg 0\) for which the visual
signal of the residual plot is very strong, and we usually do not need
any further examination of the significance of the result. However, if
the visual pattern is weak or moderate, having \(\hat{D}\) will not be
sufficient to determine if \(H_0\) should be rejected.

To address this issue while adhering to the principle of visual
inference, we can compare the estimated distance \(\hat{D}\) to the
estimated distances for the null plots in a lineup. Specifically, if a
lineup comprises 20 plots, the null hypothesis \(H_0\) will be rejected
if \(\hat{D}\) exceeds the maximum estimated distance among the
\(m - 1\) null plots, denoted as
\(\max\limits_{1 \leq i \leq m-1} {\hat{D}_{null}^{(i)}}\), where
\(\hat{D}_{null}^{(i)}\) represents the estimated distance for the
\(i\)-th null plot. This approach is equivalent to the typical lineup
protocol requiring a 95\% significance level, where \(H_0\) is rejected
if the data plot is identified as the most distinct plot by the sole
observer. The estimated distance serves as a metric to quantify the
difference between the data plot and the null plots, as intended.

For lineups consisting of more than 20 plots, the 95\% significance
level can be maintained if the number of plots is a multiple of 20.
Specifically, for lineups comprising \(20t\) plots, where \(t\) is a
positive integer, we reject \(H_0\) if \(\hat{D}\) exceeds 95\%
\({\hat{D}_{null}^{(i)}}\) for \(i = 1, \ldots, 20t-1\). The \(p\)-value
in this case is given by
\(\frac{1}{20t} + \frac{1}{20t}\sum_{i=1}^{20t-1} I\left(\hat{D}_{null}^{(i)} > \hat{D}\right)\),
where \(I(\cdot)\) is the indicator function.

Moreover, if the number of plots in a lineup, denoted by \(m\), is
sufficiently large, the empirical distribution of
\({\hat{D}_{null}^{(i)}}\) can be viewed as an approximation of the null
distribution of the estimated distance. Consequently, quantiles of the
null distribution can be estimated using the sample quantiles available
in statistical software such as R, and these quantiles can be utilized
for decision-making purposes. The details of the sample quantile
computation can be found in \citet{hyndman1996sample}. For instance, if
\(\hat{D}\) is greater than or equal to the 95\% sample quantile,
denoted as \(Q_{null}(0.95)\), we can conclude that the estimated
distance for the true residual plot is significantly different from the
estimated distance for null plots with a 95\% significance level. Based
on our experience, to obtain a stable estimate of the 95\% quantile, the
number of null plots, \(n_{null}\), typically needs to be at least 100.
However, if the null distribution exhibits a long tail, a larger number
of null plots may be required. Alternatively, a \(p\)-value can be used
to represent the probability of observing an event equally or more
extreme than the given event under the null hypothesis \(H_0\), and it
can be estimated by
\(\frac{1}{m}\sum_{i=1}^{m-1}I\left(\hat{D}_{null}^{(i)} \geq \hat{D}\right)\).

If precision in sample quantiles is not the main priority, using a
pre-calculated table of quantiles is an available option. Such a table
offers pre-determined quantiles for a specified number of observations.
It is generated by assessing numerous null plots derived from various
simulated regression models and averaging them. Essentially, this shifts
the computational burden from the user to the developer.

\subsection{Bootstrapping}\label{bootstrapping}

Bootstrap is often employed in linear regression when conducting
inference for estimated parameters. It is typically done by sampling
individual cases with replacement and refitting the regression model. If
the observed data accurately reflects the true distribution of the
population, the bootstrapped estimates can be used to measure the
variability of the parameter estimate without making strong
distributional assumptions about the data generating process.

Similarly, bootstrap can be applied on the estimated distance
\(\hat{D}\). For each refitted model \(M_{boot}^{(i)}\), there will be
an associated residual plot \(V_{boot}^{(i)}\) which can be fed into the
computer vision model to obtain \(\hat{D}_{boot}^{(i)}\), where
\(i = 1,...,n_{boot}\), and \(n_{boot}\) is the number of bootstrapped
samples. If we are interested in the variation of \(\hat{D}\), we can
use \(\hat{D}_{boot}^{(i)}\) to estimate a confidence interval.

Alternatively, since each \(M_{boot}^{(i)}\) has a set of estimated
coefficients \(\hat{\boldsymbol{\beta}}_{boot}^{(i)}\) and an estimated
variance \(\hat{\sigma^2}_{boot}^{(i)}\), a new approximated null
distribution can be construed and the corresponding 95\% sample quantile
\(Q_{boot}^{(i)}(0.95)\) can be computed. Then, if
\(\hat{D}_{boot}^{(i)} \geq Q_{boot}^{(i)}(0.95)\), \(H_0\) will be
rejected for \(M_{boot}^{(i)}\). The ratio of rejected
\(M_{boot}^{(i)}\) among all the refitted models provides an indication
of how often the assumed regression model are considered to be incorrect
if the data can be obtained repetitively from the same data generating
process. But this approach is computationally very expensive because it
requires \(n_{boot} \times n_{null}\) times of residual plot assessment.
In practice, \(Q_{null}(0.95)\) can be used to replace
\(Q_{boot}^{(i)}(0.95)\) in the computation.

\section{Data Generation}\label{data-generation}

\subsection{Simulation Scheme}\label{simulation-scheme}

While observational data is frequently employed in training models for
real-world applications, the data generating process of observational
data often remains unknown, making computation for our target variable
\(D\) unattainable. Consequently, the computer vision models developed
in this study were trained using synthetic data, including 80000
training images and 8000 test images. This approach provided us with
precise label annotations. Additionally, it ensured a large and diverse
training dataset, as we had control over the data generating process,
and the simulation of the training data was relatively cost-effective.

We have incorporated three types of residual departures of linear
regression model in the training data, including non-linearity,
heteroskedasticity and non-normality. All three departures can be
summarised by the data generating process formulated as

\begin{align}
\label{eq:data-sim}
\boldsymbol{y} &= \boldsymbol{1}_n + \boldsymbol{x}_1 + \beta_1\boldsymbol{x}_2 + \beta_2(\boldsymbol{z} + \beta_1\boldsymbol{w}) + \boldsymbol{k} \odot \boldsymbol{\varepsilon}, \\
\boldsymbol{z} &= \text{He}_j(g(\boldsymbol{x}_1, 2)), \\
\boldsymbol{w} &= \text{He}_j(g(\boldsymbol{x}_2, 2)), \\
\boldsymbol{k} &= \sqrt{\boldsymbol{1}_n + b(2 - |a|)(\boldsymbol{x}_1 + \beta_1\boldsymbol{x}_2 - a\boldsymbol{1}_n)^2},
\end{align}

\noindent where \(\boldsymbol{y}\), \(\boldsymbol{x}_1\),
\(\boldsymbol{x}_2\), \(\boldsymbol{z}\), \(\boldsymbol{w}\),
\(\boldsymbol{k}\) and \(\boldsymbol{\varepsilon}\) are vectors of size
\(n\), \(\boldsymbol{1}_n\) is a vector of ones of size \(n\),
\(\boldsymbol{x}_1\) and \(\boldsymbol{x}_2\) are two independent
predictors, \(\text{He}_j(.)\) is the \(j\)th-order probabilist's
Hermite polynomials \citep{hermite1864nouveau}, the \(\sqrt{(.)}\) and
\((.)^2\) operators are element-wise operators, \(\odot\) is the
Hadamard product, and \(g(\boldsymbol{x}, k)\) is a scaling function to
enforce the support of the random vector to be \([-k, k]^n\) defined as

\[g(\boldsymbol{x}, k) = 2k \cdot \frac{\boldsymbol{x} - x_{min}\boldsymbol{1}_n}{x_{max} - x_{min}} - k\boldsymbol{1}_n,~for~k > 0,\]
\noindent where \(x_{min} = \underset{i \in \{ 1,...,n\}}{min} x_i\),
\(x_{max} = \underset{i \in \{ 1,...,n\}}{max} x_i\) and \(x_i\) is the
\(i\)-th entry of \(\boldsymbol{x}\).

\begin{table}

\caption{\label{tab:factor}Factors used in the data generating process for synthetic data simulation. Factor $j$ and $a$ controls the non-linearity shape and the heteroskedasticity shape respectively. Factor $b$, $\sigma_\varepsilon$ and $n$ control the signal strength. Factor $\text{dist}_\varepsilon$, $\text{dist}_{x1}$ and $\text{dist}_{x2}$ specifies the distribution of $\varepsilon$, $X_1$ and $X_2$ respectively.}
\centering
\begin{tabular}[t]{ll}
\toprule
Factor & Domain\\
\midrule
j & \{2, 3, ..., 18\}\\
a & {}[-1, 1]\\
b & {}[0, 100]\\
$\beta_1$ & \{0, 1\}\\
$\beta_2$ & \{0, 1\}\\
\addlinespace
$\text{dist}_\varepsilon$ & \{discrete, uniform, normal, lognormal\}\\
$\text{dist}_{x1}$ & \{discrete, uniform, normal, lognormal\}\\
$\text{dist}_{x2}$ & \{discrete, uniform, normal, lognormal\}\\
$\sigma_{\varepsilon}$ & {}[0.0625, 9]\\
$\sigma_{X1}$ & {}[0.3, 0.6]\\
\addlinespace
$\sigma_{X2}$ & {}[0.3, 0.6]\\
n & {}[50, 500]\\
\bottomrule
\end{tabular}
\end{table}

The residuals and fitted values of the fitted model were obtained by
regressing \(\boldsymbol{y}\) on \(\boldsymbol{x}_1\). If
\(\beta_1 \neq 0\), \(\boldsymbol{x}_2\) was also included in the design
matrix. This data generation process was adapted from
\citet{li2023plot}, where it was utilized to simulate residual plots
exhibiting non-linearity and heteroskedasticity visual patterns for
human subject experiments. A summary of the factors utilized in Equation
\ref{eq:data-sim} is provided in Table \ref{tab:factor}.

In Equation \ref{eq:data-sim}, \(\boldsymbol{z}\) and \(\boldsymbol{w}\)
represent higher-order terms of \(\boldsymbol{x}_1\) and
\(\boldsymbol{x}_2\), respectively. If \(\beta_2 \neq 0\), the
regression model will encounter non-linearity issues. Parameter \(j\)
serves as a shape parameter that controls the number of tuning points in
the non-linear pattern. Typically, higher values of \(j\) lead to an
increase in the number of tuning points, as illustrated in Figure
\ref{fig:different-j}.

\begin{figure}[!h]

{\centering \includegraphics[width=1\linewidth]{paper_files/figure-latex/different-j-1} 

}

\caption{Non-linearity forms generated for the synthetic data simulation. The 17 shapes are generated by varying the order of polynomial given by $j$ in $He_j(.)$.}\label{fig:different-j}
\end{figure}

Additionally, Scaling factor \(\boldsymbol{k}\) directly affects the
error distribution and it is correlated with \(\boldsymbol{x}_1\) and
\(\boldsymbol{x}_2\). If \(b \neq 0\) and
\(\boldsymbol{\varepsilon} \sim N(\boldsymbol{0}_n, \sigma^2\boldsymbol{I}_n)\),
the constant variance assumption will be violated. Parameter \(a\) is a
shape parameter controlling the location of the smallest variance in a
residual plot as shown in Figure \ref{fig:different-a}.

\begin{figure}[!h]

{\centering \includegraphics[width=1\linewidth]{paper_files/figure-latex/different-a-1} 

}

\caption{Heteroskedasticity forms generated for the synthetic data simulation. Different shapes are controlled by the continuous factor $a$ between -1 and 1. For $a = -1$, the residual plot exhibits a "left-triangle" shape. And for $a = 1$, the residual plot exhibits a "right-triangle" shape. }\label{fig:different-a}
\end{figure}

\begin{figure}[!h]

{\centering \includegraphics[width=1\linewidth]{paper_files/figure-latex/different-e-1} 

}

\caption{Non-normality forms generated for the synthetic data simulation. Four different error distributions including discrete, lognormal, normal and uniform are considered.}\label{fig:different-e}
\end{figure}

Non-normality violations arise from specifying a non-normal distribution
for \(\boldsymbol{\varepsilon}\). In the synthetic data simulation, four
distinct error distributions are considered, including discrete,
uniform, normal, and lognormal distributions, as presented in Figure
\ref{fig:different-e}. Each distribution imparts unique characteristics
to the residuals. The discrete error distribution introduces
discreteness in residuals, while the lognormal distribution typically
yields outliers. Uniform error distribution may result in residuals
filling the entire space of the residual plot. All of these
distributions exhibit visual distinctions from the normal error
distribution.

\begin{figure}[!h]

{\centering \includegraphics[width=1\linewidth]{paper_files/figure-latex/different-j-x2-1} 

}

\caption{Residual plots of multiple linear regression models with non-linearity issues. The 17 shapes are generated by varying the order of polynomial given by $j$ in $He_j(.)$. A second predictor $\boldsymbol{x}_2$ is introduced to the regression model to create complex shapes.}\label{fig:different-j-x2}
\end{figure}

\begin{figure}[!h]

{\centering \includegraphics[width=1\linewidth]{paper_files/figure-latex/different-j-heter-1} 

}

\caption{Residual plots of models violating both the non-linearity and the heteroskedasticity assumptions. The 17 shapes are generated by varying the order of polynomial given by $j$ in $He_j(.)$, and the "left-triangle" shape is introduced by setting $a = -1$.}\label{fig:different-j-heter}
\end{figure}

\begin{figure}[!h]

{\centering \includegraphics[width=1\linewidth]{paper_files/figure-latex/different-e-heter-1} 

}

\caption{Residual plots of models violating both the non-normality and the heteroskedasticity assumptions. The four shapes are generated by using four different error distributions including discrete, lognormal, normal and uniform, and the "left-triangle" shape is introduced by setting $a = -1$. }\label{fig:different-e-heter}
\end{figure}

Equation \ref{eq:data-sim} accommodates the incorporation of the second
predictor \(\boldsymbol{x}_2\). Introducing it into the data generation
process by setting \(\beta_1 = 1\) significantly enhances the complexity
of the shapes, as illustrated in Figure \ref{fig:different-j-x2}. In
comparison to Figure \ref{fig:different-j}, Figure
\ref{fig:different-j-x2} demonstrates that the non-linear shape
resembles a surface rather than a single curve. This augmentation can
facilitate the computer vision model in learning visual patterns from
residual plots of the multiple linear regression model.

In real-world analysis, it's not uncommon to encounter instances where
multiple model violations coexist. In such cases, the residual plots
often exhibit a mixed pattern of visual anomalies corresponding to
different types of model violations. Figure \ref{fig:different-j-heter}
and \ref{fig:different-e-heter} show the visual patterns of models with
multiple model violations.

\subsection{Scagnostics}\label{scagnostics}

In Section \ref{introduction}, we mentioned that scagnostics consist of
a set of manually designed visual feature extraction functions. While
our computer vision model will learn its own feature extraction function
during training, leveraging additional information from scagnostics can
enhance the model's predictive accuracy.

For each generated residual plot, we computed four scagnostics --
``Monotonic'', ``Sparse'', ``Splines'', and ``Striped'' -- using the
\texttt{cassowaryr} R package \citep{mason2022cassowaryr}. These
computed measures, along with the number of observations from the fitted
model, were provided as the second input for the computer vision model.
While other scagnostics provide valuable insights, they come with high
computational costs and are not suitable for quick inference.

\subsection{Balanced Dataset}\label{balanced-dataset}

To train a robust computer vision model, we deliberately controlled the
distribution of the target variable \(D\) in the training data. We
ensured that it followed a uniform distribution between \(0\) and \(7\).
This was achieved by organizing \(50\) buckets, each exclusively
accepting training samples with \(D\) falling within the range
\([7(i - 1)/49, 7i/49)\) for \(i < 50\), where \(i\) represents the
index of the \(i\)-th bucket. For the \(50\)-th bucket, any training
samples with \(D \geq 7\) were accepted.

With 80000 training images prepared, each bucket accommodated a maximum
of \(80000 \div 50 = 1600\) training samples. The simulator iteratively
sampled parameter values from the parameter space, generated residuals
and fitted values using the data generation process, computed the
distance, and checked if the sample fitted within the corresponding
bucket. This process continued until all buckets were filled.

Similarly, we adopted the same methodology to prepare 8000 test images
for performance evaluation and model diagnostics.

\section{Model Architecture}\label{model-architecture}

\begin{figure}

{\centering \includegraphics[width=1\linewidth]{cnn} 

}

\caption{Diagram of the architecture of the optimized computer vision model. Numbers at the bottom of each box show the shape of the output of each layer. The band of each box drawn in a darker colour indicates the use of the rectified linear unit activation function.  Yellow boxes are 2D convolutional layers, orange boxes are pooling layers, the grey box is the concatenation layer, and the purple boxes are dense layers.}\label{fig:cnn-diag}
\end{figure}

The architecture of the computer vision model is adapted from a
well-established architecture known as VGG16, which has demonstrated
high performance in image classification \citep{simonyan2014very}.
Figure \ref{fig:cnn-diag} provides a diagram of the architecture. More
details about the neural network layers used in this study are provided
in the Appendix.

The model begins with an input layer of shape
\(n \times h \times w \times 3\), capable of handling \(n\) RGB images.
This is followed by a grayscale conversion layer utilizing the luma
formula under the Rec. 601 standard, which converts the colour image to
grayscale. Grayscale suffices for our task since data points are plotted
in black. We experiment with three combinations of \(h\) and \(w\):
\(32 \times 32\), \(64 \times 64\), and \(128 \times 128\), aiming to
achieve sufficiently high image resolution for the problem at hand.

The processed image is used as the input for the first convolutional
block. The model comprises at most five consecutive convolutional
blocks, mirroring the original VGG16 architecture. Within each block,
there are two 2D convolutional layers followed by two activation layers,
respectively. Subsequently, a 2D max-pooling layer follows the second
activation layer. The 2D convolutional layer convolves the input with a
fixed number of \(3 \times 3\) convolution filters, while the 2D
max-pooling layer downsamples the input along its spatial dimensions by
taking the maximum value over a \(2 \times 2\) window for each channel
of the input. The activation layer employs the rectified linear unit
(ReLU) activation function, a standard practice in deep learning, which
introduces a non-linear transformation of the output of the 2D
convolutional layer. Additionally, to regularize training, a batch
normalization layer is added after each 2D convolutional layer and
before the activation layer. Finally, a dropout layer is appended at the
end of each convolutional block to randomly set some inputs to zero
during training, further aiding in regularization.

The output of the last convolutional block is summarized by either a
global max pooling layer or a global average pooling layer, resulting in
a two-dimensional tensor. To leverage the information contained in
scagnostics, this tensor is concatenated with an additional
\(n \times 5\) tensor, which contains the ``Monotonic,'' ``Sparse,''
``Splines,'' and ``Striped'' measures, along with the number of
observations for \(n\) residual plots.

The concatenated tensor is then fed into the final prediction block.
This block consists of two fully-connected layers. The first layer
contains at least \(128\) units, followed by a dropout layer.
Occasionally, a batch normalization layer is inserted between the
fully-connected layer and the dropout layer for regularization purposes.
The second fully-connected layer consists of only one unit, serving as
the output of the model.

The model weights \(\boldsymbol{\theta}\) were randomly initialized and
they were optimized by the Adam optimizer with the mean square error
loss function

\[\hat{\boldsymbol{\theta}} = \underset{\boldsymbol{\theta}}{argmin}\frac{1}{n_{train}}\sum_{i=1}^{n_{train}}(D_i - f_{\boldsymbol{\theta}}(V_i, S_i))^2,\]

\noindent where \(n_{train}\) is the number of training samples, \(V_i\)
is the \(i\)-th residual plot and \(S_i\) is the additional information
about the \(i\)-th residual plot including four scagnostics and the
number of observations.

\section{Model Training}\label{model-training}

To achieve a near-optimal deep learning model, hyperparameters like the
learning rate often need to be fine-tuned using a tuner. In our study,
we utilized the Bayesian optimization tuner from the \texttt{KerasTuner}
Python library \citep{omalley2019kerastuner} for this purpose. A
comprehensive list of hyperparameters is provided in Table
\ref{tab:hyperparameter}.

The ``number of base filters'' determines the number of filters for the
first 2D convolutional layer. In the VGG16 architecture, the number of
filters for the 2D convolutional layer in a block is typically twice the
number in the previous block, except for the last block, which maintains
the same number of convolution filters as the previous one. This
hyperparameter aids in controlling the complexity of the computer vision
model. Higher numbers of base filters result in more trainable
parameters. Likewise, the ``number of units for the fully-connected
layer'' determines the complexity of the final prediction block.
Increasing the number of units enhances model complexity, resulting in
more trainable parameters.

The dropout rate and batch normalization are flexible hyperparameters
that work in conjunction with other parameters to facilitate smooth
training. A higher dropout rate is necessary when the model tends to
overfit the training dataset by learning too much noise. Conversely, a
lower dropout rate is preferred when the model is complex and
challenging to converge. Batch normalization, on the other hand,
addresses the internal covariate shift problem arising from the
randomness in weight initialization and input data. It helps stabilize
and accelerate the training process by normalizing the activations of
each layer.

Additionally, incorporating additional inputs such as scagnostics and
the number of observations can potentially enhance prediction accuracy.
Therefore, we allowed the tuner to determine whether these inputs were
necessary for optimal model performance.

The learning rate is a crucial hyperparameter, as it dictates the step
size of the optimization algorithm. A high learning rate can help the
model avoid local minima but risks overshooting and missing the global
minimum. Conversely, a low learning rate smoothens the training process
but makes the convergence time longer and increases the likelihood of
getting trapped in local minima.

Our model was trained on the MASSIVE M3 high-performance computing
platform \citep{goscinski2014multi}, using TensorFlow
\citep{abadi2016tensorflow} and Keras \citep{chollet2015keras}. During
training, 80\% of the training data was utilized for actual training,
while the remaining 20\% was used as validation data. The Bayesian
optimization tuner conducted 100 trials to identify the best
hyperparameter values based on validation root mean square error. The
tuner then restored the best epoch of the best model from the trials.
Additionally, we applied early stopping, terminating the training
process if the validation root mean square error fails to improve for 50
epochs. The maximum allowed epochs was set at 2000, although no models
reached this threshold.

\begin{table}

\caption{\label{tab:hyperparameter}Name of hyperparameters and their correspoding domain for the computer vision model.}
\centering
\begin{tabular}[t]{ll}
\toprule
Hyperparameter & Domain\\
\midrule
Number of base filters & \{4, 8, 16, 32, 64\}\\
Dropout rate for convolutional blocks & {}[0.1, 0.6]\\
Batch normalization for convolutional blocks & \{false, true\}\\
Type of global pooling & \{max, average\}\\
Ignore additional inputs & \{false, true\}\\
\addlinespace
Number of units for the fully-connected layer & \{128, 256, 512, 1024, 2048\}\\
Batch normalization for the fully-connected layer & \{false, true\}\\
Dropout rate for the fully-connected layer & {}[0.1, 0.6]\\
Learning rate & {}[1e-8, 1e-1]\\
\bottomrule
\end{tabular}
\end{table}

Based on the tuning process described above, the optimized
hyperparameter values are presented in Table
\ref{tab:best-hyperparameter}. It was observable that a minimum of
\(32\) base filters was necessary, with the preferable choice being
\(64\) base filters for both the \(64 \times 64\) and \(128 \times 128\)
models, mirroring the original VGG16 architecture. The optimized dropout
rate for convolutional blocks hovered around \(0.4\), and incorporating
batch normalization for convolutional blocks proved beneficial for
performance.

All optimized models chose to retain the additional inputs, contributing
to the reduction of validation error. The number of units required for
the fully-connected layer was \(256\), a relatively modest number
compared to the VGG16 classifier, suggesting that the problem at hand
was less complex. The optimized learning rates were smaller than the
default value recommended by Keras, which is 0.001.

\begin{table}

\caption{\label{tab:best-hyperparameter}Hyperparameters values for the optimized computer vision models with different input sizes.}
\centering
\resizebox{\linewidth}{!}{
\begin{tabular}[t]{llll}
\toprule
Hyperparameter & $32 \times 32$ & $64 \times 64$ & $128 \times 128$\\
\midrule
Number of base filters & 32 & 64 & 64\\
Dropout rate for convolutional blocks & 0.4 & 0.3 & 0.4\\
Batch normalization for convolutional blocks & true & true & true\\
Type of global pooling & max & average & average\\
Ignore additional inputs & false & false & false\\
\addlinespace
Number of units for the fully-connected layer & 256 & 256 & 256\\
Batch normalization for the fully-connected layer & false & true & true\\
Dropout rate for the fully-connected layer & 0.2 & 0.4 & 0.1\\
Learning rate & 0.0003 & 0.0006 & 0.0052\\
\bottomrule
\end{tabular}}
\end{table}

\section{Results}\label{results}

\subsection{Model Performance}\label{model-performance}

The training and test performance for the optimized models with three
different input sizes are summarized in Table \ref{tab:performance}.
Among these models, the \(64 \times 64\) model and the \(32 \times 32\)
model consistently exhibited the best training and test performance,
respectively. The mean absolute error indicated that the difference
between \(\hat{D}\) and \(D\) was approximately \(0.43\) on the test
set, a negligible deviation considering the normal range of \(D\)
typically falls between \(0\) and \(7\). The high \(R^2\) values also
suggested that the predictions were largely linearly correlated with the
target.

Figure \ref{fig:model-performance} presents a hexagonal heatmap for
\(D - \hat{D}\) versus \(\hat{D}\). The red smoothing curves, fitted by
generalized additive models \citep{hastie2017generalized}, demonstrate
that all the optimized models perform admirably on both the training and
test sets. No structural issues are noticeable in Figure
\ref{fig:model-performance}, but some minor issues regarding
over-prediction and under-prediction are observed.

The figure highlights that most under-predictions occurred when
\(\hat{D} < 3\), while over-predictions occurred predominantly when
\(3 < \hat{D} < 6\). For input images representing null plots, it was
expected that the models will over-predict the distance, as explained in
Section \ref{lineup-evaluation}. However, it can not explain the
under-prediction issue. Therefore, we analysed the relationship between
residuals and all the factors involved in the data generating process.
We found that most issues actually arose from non-linearity problems and
the presence of a second predictor in the regression model as
illustrated in Figure \ref{fig:over-under}. When the variance for the
error distribution was small, the optimized model tended to
under-predict the distance. Conversely, when the error distribution had
a large variance, the model tended to over-predict the distance.

Since most of the deviation stemmed from the presence of non-linearity
violations, to further investigate this, we split the test set based on
violation types and re-evaluated the performance, as detailed in Table
\ref{tab:performance-sub}. It was evident that metrics for null plots
were notably worse compared to other categories. Furthermore, residual
plots solely exhibiting non-normality issues were the easiest to
predict, with very low test root mean square error (RMSE) at around
\(0.3\). Residual plots with non-linearity issues were more challenging
to assess than those with heteroskedasticity or non-normality issues.
Assessing residual plots with heteroskedasticity was not as difficult as
assessing those with non-linearity issues. When multiple violations were
introduced to a residual plot, the performance metrics typically lay
between the metrics for each individual violation.

Based on the model performance metrics, we chose to use the
best-performing model evaluated on the test set, namely the
\(32 \times 32\) model, for the subsequent analysis.

\begin{table}

\caption{\label{tab:performance}The training and test performance of three optimized models with different input sizes.}
\centering
\begin{tabular}[t]{lllll}
\toprule
 & RMSE & $R^2$ & MAE & Huber loss\\
\midrule
\addlinespace[0.3em]
\multicolumn{5}{l}{\textbf{Training set}}\\
\hspace{1em}$32 \times 32$ & 0.531 & 0.937 & 0.364 & 0.126\\
\hspace{1em}$64 \times 64$ & 0.405 & 0.963 & 0.260 & 0.072\\
\hspace{1em}$128 \times 128$ & 0.432 & 0.959 & 0.290 & 0.084\\
\addlinespace[0.3em]
\multicolumn{5}{l}{\textbf{Test set}}\\
\hspace{1em}$32 \times 32$ & 0.660 & 0.901 & 0.434 & 0.181\\
\hspace{1em}$64 \times 64$ & 0.674 & 0.897 & 0.438 & 0.186\\
\hspace{1em}$128 \times 128$ & 0.692 & 0.892 & 0.460 & 0.199\\
\bottomrule
\end{tabular}
\end{table}

\begin{figure}[!h]

{\centering \includegraphics[width=1\linewidth]{paper_files/figure-latex/model-performance-1} 

}

\caption{Hexagonal heatmap for residuals vs predicted distance on training and test data for three optimized models with different input sizes. The brown lines are smoothing curves produced by fitting gnealized additive models. The area over the zero line in light yellow indicates under-prediction, and the area under the zero line in light green indicates over-prediction.}\label{fig:model-performance}
\end{figure}

\begin{figure}[!h]

{\centering \includegraphics[width=1\linewidth]{paper_files/figure-latex/over-under-1} 

}

\caption{Scatter plots for residuals vs $\sigma$ on test data for the $32 \times 32$ optimized model. The data is grouped by whether the regression has only non-linearity violation, and whether it includes a second predictor in the regression formula. The brown lines are smoothing curves produced by fitting gnealized additive models. The area over the zero line in light yellow indicates under-prediction, and the area under the zero line in light green indicates over-prediction.}\label{fig:over-under}
\end{figure}

\begin{table}

\caption{\label{tab:performance-sub}The training and test performance of the $32 \times 32$ model presented with different model violations.}
\centering
\resizebox{\linewidth}{!}{
\begin{tabular}[t]{lllll}
\toprule
Violations & \#training samples & Training RMSE & \#test samples & Test RMSE\\
\midrule
no violations & 1546 & 1.149 & 155 & 1.267\\
non-linearity & 22184 & 0.625 & 2218 & 0.787\\
heteroskedasticity & 10826 & 0.528 & 1067 & 0.602\\
non-linearity + heteroskedasticity & 10221 & 0.593 & 985 & 0.751\\
non-normality & 10797 & 0.279 & 1111 & 0.320\\
\addlinespace
non-linearity + non-normality & 8835 & 0.472 & 928 & 0.600\\
heteroskedasticity + non-normality & 7870 & 0.354 & 819 & 0.489\\
non-linearity + heteroskedasticity + non-normality & 7721 & 0.432 & 717 & 0.620\\
\bottomrule
\end{tabular}}
\end{table}

\subsection{Comparison with Human Visual Inference and Conventional
Tests}\label{comparison-with-human-visual-inference-and-conventional-tests}

\subsubsection{Overview of the Human Subject
experiment}\label{overview-of-the-human-subject-experiment}

In order to check the validity of the proposed computer vision model,
residual plots presented in the human subject experiment conducted by
\citet{li2023plot} will be assessed.

This study has collected \(7974\) human responses to \(1152\) lineups.
Each lineup contains one randomly placed true residual plot and 19 null
plots. Among the \(1152\) lineups, \(24\) are attention check lineups in
which the visual patterns are designed to be extremely obvious and very
different from the corresponding to null plots, \(36\) are null lineups
where all the lineups consist of only null plots, \(279\) are lineups
with uniform predictor distribution evaluated by \(11\) participants,
and the remaining \(813\) are lineups with discrete, skewed or normal
predictor distribution evaluated by \(5\) participants. Attention check
lineups and null lineups will not be assessed in the following analysis.

In \citet{li2023plot}, the residual plots are simulated from a data
generating process which is a special case of Equation
\ref{eq:data-sim}. The main characteristic is the model violations are
introduced separately, meaning non-linearity and heteroskedasticity will
not co-exist in one lineup but assigned uniformly to all lineups.
Additionally, non-normality and multiple predictors are not considered
in the experimental design.

\subsubsection{Model performance on the human
data}\label{model-performance-on-the-human-data}

\begin{table}

\caption{\label{tab:experiment-performance}The performance of the $32 \times 32$ model on the data used in the human subject experiment.}
\centering
\begin{tabular}[t]{lllll}
\toprule
Voilation & RMSE & $R^2$ & MAE & Huber loss\\
\midrule
heteroskedasticity & 0.721 & 0.852 & 0.553 & 0.235\\
non-linearity & 0.738 & 0.770 & 0.566 & 0.246\\
\bottomrule
\end{tabular}
\end{table}

For each lineup used in \citet{li2023plot}, there is one true residual
plot and 19 null plots. While the distance \(D\) for the true residual
plot depends on the underlying data generating process, the distance
\(D\) for the null plots is zero. We have used our optimized computer
vision model to estimate distance for both the true residual plots and
the null plots. To have a fair comparison, \(H_0\) will be rejected if
the true residual plot has the greatest approximated distance among all
plots in a lineup. Additionally, the appropriate conventional tests
including the Ramsey Regression Equation Specification Error Test
(RESET) \citep{ramsey1969tests} for non-linearity and the Breusch-Pagan
test \citep{breusch1979simple} for heteroskedasticity were applied on
the same data for comparison.

The performance metrics of \(\hat{D}\) for true residual plots are
outlined in Table \ref{tab:experiment-performance}. It's notable that
all performance metrics are slightly worse than those evaluated on the
test data. Nevertheless, the mean absolute error remains at a low level,
and the linear correlation between the prediction and the true value
remains very high. Lineups with non-linearity issues are more
challenging to predict than those with heteroskedasticity issues.

Table \ref{tab:human-conv-table} provides a summary of the agreement
between decisions made by the computer vision model and conventional
tests. The agreement rates between conventional tests and the computer
vision model are 85.95\% and 79.69\% for residual plots containing
heteroskedasticity and non-linearity patterns, respectively. These
figures are higher than those calculated for visual tests conducted by
human, indicating that the computer vision model exhibits behaviour more
akin to the best available conventional tests. However, Figure
\ref{fig:conv-mosaic} shows that the computer vision model does not
always reject when the conventional tests reject. And a small number of
plots will be rejected by computer vision model but not by conventional
tests. This suggests that conventional tests are more sensitive than the
computer vision model.

Figure \ref{fig:pcp} further illustrates the decisions made by visual
tests conducted by human, computer vision models, and conventional
tests, using a parallel coordinate plots. It can be observed that all
three tests will agree with each other for around 50\% of the cases.
When visual tests conducted by human do not reject, there are
substantial amount of cases where computer vision model also not reject
but conventional tests reject. There are much fewer cases that do not
reject by visual tests and conventional tests, but reject by computer
vision models. This indicates computer vision model can behave like
visual tests conducted by human better than conventional tests.
Moreover, there are great proportion of cases where visual tests
conducted by human is the only test who does not reject, .

When plotting the decision against the distance, as illustrated in
Figure \ref{fig:power}, several observations emerge. Firstly, for the
computer vision model, when the distance \(D > 4\), almost all residual
plots are rejected. However, for visual tests conducted by humans, the
threshold is higher, set at \(D > 5\). And conventional tests exhibit
some non-rejection cases for \(D > 3\). Moreover, the computer vision
model tends to have fewer rejected cases than conventional tests when
\(D < 2\). However, visual tests demonstrate the lowest sensitivity to
residual plots with small distances. Therefore, tests based on the
computer vision model are less sensitive to small deviations from model
assumptions than conventional tests but more sensitive to moderate
deviations. Rejection decisions are fitted by logistic regression models
with no intercept terms but an offset equals to
\(\text{log}(0.05/0/95)\). The fitting curves for the computer vision
model fall between those of conventional tests and visual tests for both
non-linearity and heteroskedasticity. There is still potential to refine
the computer vision model to better align its behaviour with visual
tests.

In the experiment conducted in \citet{li2023plot}, participants were
allowed to make multiple selections for a lineup. The weighted detection
rate is computed by assigning weights to each detection. If the
participant selects zero plots, the corresponding weight is 0.05;
otherwise, the weight is 1 divided by the number of selections. The
\(\delta\)-difference is defined by \citet{chowdhury2018measuring} as

\begin{equation}
\delta = \hat{D} - \underset{j}{max}\left(\hat{D}_{null}^{(j)}\right) \quad \text{for}~j = 1,...,m-1,
\end{equation}

\noindent where \(\hat{D}_{null}^{(j)}\) is the approximated distance
for the \(j\)-th null plot, and \(m\) is the number of plots in a
lineup.

Figure \ref{fig:delta} displays the scatter plot of the weighted
detection rate vs the \(\delta\)-difference. It indicates that the
weighted detection rate increases as the \(\delta\)-difference
increases, particularly when the \(\delta\)-difference is greater than
zero. A negative \(\delta\)-difference suggests that there is at least
one null plot in the lineup with a stronger visual signal than the true
residual plot. In some instances, the weighted detection rate is close
to one, yet the \(\delta\)-difference is negative. This discrepancy
implies that the distance measure, or the estimated distance, may not
perfectly reflect actual human behaviour.

\begin{table}

\caption{\label{tab:human-conv-table}Summary of the comparison of decisions made by computer vision model with decisions made by conventional tests and visual tests conducted by human.}
\centering
\begin{tabular}[t]{llll}
\toprule
Violations & \#Samples & \#Agreements & Agreement rate\\
\midrule
\addlinespace[0.3em]
\multicolumn{4}{l}{\textbf{Compared with conventional tests}}\\
\hspace{1em}heteroskedasticity & 540 & 464 & 0.8593\\
\hspace{1em}non-linearity & 576 & 459 & 0.7969\\
\addlinespace[0.3em]
\multicolumn{4}{l}{\textbf{Compared with visual tests conducted by human}}\\
\hspace{1em}heteroskedasticity & 540 & 367 & 0.6796\\
\hspace{1em}non-linearity & 576 & 385 & 0.6684\\
\bottomrule
\end{tabular}
\end{table}

\begin{figure}[!h]

{\centering \includegraphics[width=1\linewidth]{paper_files/figure-latex/conv-mosaic-1} 

}

\caption{Rejection rate ($p$-value $\leq0.05$) of computer vision models conditional on conventional tests on non-linearity (left) and heteroskedasticity (right) lineups displayed using a mosaic plot.}\label{fig:conv-mosaic}
\end{figure}

\begin{figure}[!h]

{\centering \includegraphics[width=1\linewidth]{paper_files/figure-latex/power-1} 

}

\caption{Comparison of power of visual tests, conventional tests and the computer vision model. Marks along the x-axis at the bottom of the plot represent rejections made by each type of test. Marks at the top of the plot represent acceptances. Power curves are fitted by logistic regression models with no intercept but an offset equals to $\text{log}(0.05/0.95)$.}\label{fig:power}
\end{figure}

\begin{figure}

{\centering \includegraphics[width=1\linewidth]{paper_files/figure-latex/pcp-1} 

}

\caption{Parallel coordinate plots of decisions made by computer vision model, conventional tests and visual tests made by human.}\label{fig:pcp}
\end{figure}

\begin{figure}[!h]

{\centering \includegraphics[width=1\linewidth]{paper_files/figure-latex/delta-1} 

}

\caption{A weighted detection rate vs $\delta$-differnence plot. The brown line is smoothing curve produced by fitting gnealized additive models}\label{fig:delta}
\end{figure}

\section{Examples}\label{examples}

In this section, we will present the performance of trained computer
vision model on three example datasets. These include the dataset
associated with the residual plot displaying a ``left-triangle'' shape,
as displayed in Figure \ref{fig:false-finding}, along with the Boston
housing dataset, and the ``dino'' datasets from the \texttt{datasaurus}
R package.

The first example illustrates a scenario where both the computer vision
model and human visual inspection successfully avoid rejecting \(H_0\)
when \(H_0\) is true, contrary to conventional tests. This underscores
the necessity of visually examining the residual plot.

In the second example, we encounter a more pronounced violation of the
model, resulting in rejection of \(H_0\) by all three tests. This
highlights the practicality of the computer vision model, particularly
for less intricate tasks.

The third example presents a situation where the model deviation is
non-typical. Here, the computer vision model and human visual inspection
reject \(H_0\), whereas some commonly used conventional tests do not.
This emphasizes the benefits of visual inspection and the unique
advantage of the computer vision model, which, like humans, makes
decisions based on visual discoveries.

Additionally, we will outline the workflow for employing this model with
the \texttt{autovi} R package.

\subsection{Left-triangle}\label{left-triangle}

In Section \ref{introduction}, we presented an example residual plot
showcased in Figure \ref{fig:false-finding}, illustrating how humans
might misinterpret the ``left-triangle'' shape as indicative of
heteroskedasticity. Additionally, the Breusch-Pagan test yielded a
rejection with a \(p\)-value of 0.046, despite the residuals originating
from a correctly specified model. Figure \ref{fig:false-lineup} offers a
lineup for this fitted model, showcasing various degrees of
``left-triangle'' shape across all residual plots. This phenomenon is
evidently caused by the skewed distribution of the fitted values.
Notably, if the residual plot were evaluated through a visual test, it
would not be rejected since the true residual plot positioned at 10 can
not be distinguished from the others.

Moving forward, we will employ the trained computer vision model to
assess the residual plot with the assistance of the \texttt{autovi} R
package. This package is accessible on the GitHub repository
\texttt{TengMCing/autovi}, and users can install it by executing the
command \texttt{remotes::install\_github("TengMCing/autovi")} in R.

The results of the \texttt{list\_keras\_model()} function provide a list
of available trained computer vision models. To perform the residual
plot assessment, users need to supply a fitted linear regression model
and select a trained Keras model to initialize the checker.

\begin{Shaded}
\begin{Highlighting}[]
\NormalTok{checker }\OtherTok{\textless{}{-}} \FunctionTok{auto\_vi}\NormalTok{(}\AttributeTok{fitted\_mod =}\NormalTok{ mod, }
                   \AttributeTok{keras\_mod =} \FunctionTok{get\_keras\_model}\NormalTok{(}\StringTok{"vss\_phn\_32"}\NormalTok{))}
\end{Highlighting}
\end{Shaded}

When invoking the \texttt{check()} method to execute the assessment,
users can specify the number of null plots and bootstrapped samples
using the \texttt{null\_draws} and \texttt{boot\_draws} arguments,
respectively. The diagnostic results can be viewed by directly printing
the object. Alternatively, a summary plot of the assessment can be
generated using the \texttt{summary\_plot()} method.

\begin{Shaded}
\begin{Highlighting}[]
\NormalTok{checker}\SpecialCharTok{$}\FunctionTok{check}\NormalTok{(}\AttributeTok{null\_draws =} \DecValTok{200}\DataTypeTok{L}\NormalTok{, }\AttributeTok{boot\_draws =} \DecValTok{200}\DataTypeTok{L}\NormalTok{)}
\NormalTok{checker}\SpecialCharTok{$}\FunctionTok{summary\_plot}\NormalTok{()}
\end{Highlighting}
\end{Shaded}

Figure \ref{fig:false-check} presents the results of the assessment by
the computer vision model. Notably, the observed visual signal strength
is considerably lower than the 95\% sample quantile of the null
distribution. Moreover, the bootstrapped distribution suggests that it
is highly improbable for the fitted model to be misspecified as the
majority of bootstrapped fitted models will not be rejected. Thus, for
this particular fitted model, both the visual test and the computer
vision model will not reject \(H_0\). However, the Breusch-Pagan test
will reject \(H_0\) because it can not effectively utilize information
from null plots.

The attention map at plot B of Figure \ref{fig:false-check} suggests
that the estimation is highly influenced by the top-right and
bottom-right part of the residual plot, as it forms two vertices of the
triangular shape. A principle component analysis is also performed on
the output of the global pooling layer of the computer vision model. As
mentioned in \citet{simonyan2014very}, a computer vision model built
upon the convolutional blocks can be viewed as a feature extractor. For
the \(32 \times 32\) model, there are 256 features outputted from the
global pooling layer, which can be further used for different visual
tasks not limited to distance prediction. To see if these features can
be effectively used for distinguishing null plots and true residual
plot, we linearly project them into the PC1 vs PC2 space as shown in
Figure \ref{fig:false-check}. It can be observed that because the
bootstrapped plots are mostly similar to the null plots, the points
drawn in different colours are mixed together. The true residual plot is
also covered by both the cluster of null plots and cluster of
bootstrapped plots. This accurately reflects our understanding of Figure
\ref{fig:false-lineup}.

\begin{figure}[!h]

{\centering \includegraphics[width=1\linewidth]{paper_files/figure-latex/false-check-1} 

}

\caption{A summary of the residual plot assessment evaluted on 200 null plots and 200 bootstrapped plots. (A) The true residual plot exhibiting a "left-triangle" shape. (B) The attention map produced by computing the gradient of the output with respect to the greyscale input.  (C) The density plot of estimated distnace for null plots and bootstrapped plots. The blue area indicates the distribution of estimated distances for bootstrapped plots, while the yellow area represents the distribution of estimated distances for null plots. The fitted model will not be rejected since $\hat{D} < Q_{null}(0.95)$. (D) PC2 vs PC1 plot for features extracted from the global pooling layer of the computer vision model.  }\label{fig:false-check}
\end{figure}

\begin{figure}[!h]

{\centering \includegraphics[width=1\linewidth]{paper_files/figure-latex/false-lineup-1} 

}

\caption{A lineup of residual plots displaying "left-triangle" visual patterns. The true residual plot occupies position 10, yet there are no discernible visual patterns that distinguish it from the other plots.}\label{fig:false-lineup}
\end{figure}

\subsection{Boston housing}\label{boston-housing}

The Boston housing dataset, originally published by
\citet{harrison1978hedonic}, offers insights into housing in the Boston,
Massachusetts area. For illustration purposes, we will utilize a reduced
version from Kaggle, comprising 489 rows and 4 columns: average number
of rooms per dwelling (RM), percentage of lower status of the population
(LSTAT), pupil-teacher ratio by town (PTRATIO), and Median value of
owner-occupied homes in \$1000's (MEDV). In our analysis, MEDV will
serve as the response variable, while the other columns will function as
predictors in a linear regression model. Our primary focus is to detect
non-linearity, because the relationships between RM and MEDV or LSTAT
and MEDV are non-linear.

Figure \ref{fig:boston-check} displays the residual plot and the
assessment conducted by the computer vision model. A clear non-linearity
pattern resembling a ``U'' shape is shown in the plot A. Furthermore,
the RESET test yields a very small \(p\)-value. The estimated distance
\(\hat{D}\) significantly exceeds \(Q_{null}(0.95)\), leading to
rejection of \(H_0\). The bootstrapped distribution also suggests that
almost all the bootstrapped fitted models will be rejected, indicating
that the fitted model is unlikely to be correctly specified. The
attention map in plot B suggests the center of the image has higher
leverage than other areas, and it is the turning point of the ``U''
shape. The principal component analysis provided in plot D shows two
distinct clusters of data points, further underlines the visual
differences between bootstrapped plots and null plots. This coincides
the findings from Figure \ref{fig:boston-lineup}, where the true plot
exhibiting a ``U'' shape is visually distinctive from null plots. If a
visual test is conducted by human, \(H_0\) will also be rejected.

\begin{figure}[!h]

{\centering \includegraphics[width=1\linewidth]{paper_files/figure-latex/boston-check-1} 

}

\caption{A summary of the residual plot assessment for the Boston housing fitted model evaluted on 200 null plots and 200 bootstrapped plots. (A) The true residual plot exhibiting a "U" shape. (B) The attention map produced by computing the gradient of the output with respect to the greyscale input.  (C) The density plot of estimated distnace for null plots and bootstrapped plots. The blue area indicates the distribution of estimated distances for bootstrapped plots, while the yellow area represents the distribution of estimated distances for null plots. The fitted model will be rejected since $\hat{D} \geq Q_{null}(0.95)$. (D) PC2 vs PC1 plot for features extracted from the global pooling layer of the computer vision model. }\label{fig:boston-check}
\end{figure}

\begin{figure}[!h]

{\centering \includegraphics[width=1\linewidth]{paper_files/figure-latex/boston-lineup-1} 

}

\caption{A lineup of residual plots for the Boston housing fitted model. The true residual plot is at position 7. It can be easily identified as the most different plot.}\label{fig:boston-lineup}
\end{figure}

\subsection{Datasaurus}\label{datasaurus}

The computer vision model possesses the capability to detect not only
typical issues like non-linearity, heteroskedasticity, and non-normality
but also artifact visual patterns resembling real-world objects, as long
as they do not appear in null plots. These visual patterns can be
challenging to categorize in terms of model violations. Therefore, we
will employ the RESET test, the Breusch-Pagan test, and the Shapiro-Wilk
test \citep{shapiro1965analysis} for comparison.

The ``dino'' dataset within the \texttt{datasaurus} R package
exemplifies this scenario. With only two columns, x and y, fitting a
regression model to this data yields a residual plot resembling a
``dinosaur,'' as displayed in plot A of Figure \ref{fig:dino-check}.
Unsurprisingly, this distinct residual plot stands out in a lineup, as
shown in Figure \ref{fig:dino-lineup}. A visual test conducted by humans
would undoubtedly reject \(H_0\).

According to the residual plot assessment by the computer vision model,
\(\hat{D}\) exceeds \(Q_{null}(0.95)\), warranting a rejection of
\(H_0\). Additionally, most of the bootstrapped fitted models will be
rejected, indicating an misspecified model. However, both the RESET test
and the Breusch-Pagan test yield \(p\)-values greater than 0.3, leading
to a non-rejection of \(H_0\). Only the Shapiro-Wilk test rejects the
normality assumption with a small \(p\)-value.

More importantly, the attention map in plot B of Figure
\ref{fig:dino-check} clearly exhibits a ``dinosaur'' shape, strongly
suggesting the prediction of the distance is based on human perceptible
visual patterns. The computer vision model is also capable of capturing
the contour or the outline of the embedded shape, just like human being
reading residual plots. The principle component analysis in plot D of
Figure \ref{fig:dino-check} also shows that the cluster of bootstrapped
plots is at the corner of the cluster of null plots.

More importantly, the attention map in plot B of Figure
\ref{fig:dino-check} clearly exhibits a ``dinosaur'' shape, strongly
suggesting that the distance prediction is based on human-perceptible
visual patterns. The computer vision model effectively captures the
contour or outline of the embedded shape, similar to how humans
interpret residual plots. Additionally, the principal component analysis
in plot D of Figure \ref{fig:dino-check} demonstrates that the cluster
of bootstrapped plots is positioned at the corner of the cluster of null
plots.

In practice, without accessing the residual plot, it would be
challenging to identify the artificial pattern of the residuals.
Moreover, conducting a normality test for a fitted regression model is
not always standard practice among analysts. Even when performed,
violating the normality assumption is sometimes deemed acceptable,
especially considering the application of quasi-maximum likelihood
estimation in linear regression. This example underscores the importance
of evaluating residual plots and highlights how the proposed computer
vision model can facilitate this process.

\begin{figure}[!h]

{\centering \includegraphics[width=1\linewidth]{paper_files/figure-latex/dino-check-1} 

}

\caption{A summary of the residual plot assessment for the Datasaurus fitted model evaluted on 200 null plots and 200 bootstrapped plots. (A) The true residual plot exhibiting a "dinosaur" shape. (B) The attention map produced by computing the gradient of the output with respect to the greyscale input.  (C) The density plot of estimated distnace for null plots and bootstrapped plots. The blue area indicates the distribution of estimated distances for bootstrapped plots, while the yellow area represents the distribution of estimated distances for null plots. The fitted model will be rejected since $\hat{D} \geq Q_{null}(0.95)$. (D) PC2 vs PC1 plot for features extracted from the global pooling layer of the computer vision model.}\label{fig:dino-check}
\end{figure}

\begin{figure}[!h]

{\centering \includegraphics[width=1\linewidth]{paper_files/figure-latex/dino-lineup-1} 

}

\caption{A lineup of residual plots for the fitted model on the "dinosaur" dataset. The true residual plot is at position 17. It can be easily identified as the most different plot as the visual pattern is extremly artificial.}\label{fig:dino-lineup}
\end{figure}

\section{Limitations and Future Work}\label{limitations-and-future-work}

Despite the computer vision model performing well with general cases
under the synthetic data generation scheme and the three examples used
in this paper, this study has several limitations that could guide
future work.

The proposed distance measure assumes that the true model is a classical
normal linear regression model, which can be too restrictive. Although
this paper does not address the relaxation of this assumption, there are
potential methods to evaluate other types of regression models. The most
comprehensive approach would be to define a distance measure for each
different class of regression model and then train the computer vision
model following the methodology described in this paper. To accelerate
training, one could use the convolutional blocks of our trained model as
a feature extractor and perform transfer learning on top of it, as these
blocks effectively capture shapes in residual plots. Another approach
would be to transform the residuals so they are roughly normally
distributed and have constant variance. If only raw residuals are used,
the distance-based statistical testing compares the difference in
distance to a classical normal linear regression model for the true plot
and null plots. This comparison is meaningful only if the difference can
be identified by the distance measure proposed in this paper.

There are other types of residual plots commonly used in diagnostics,
such as residuals vs.~predictor and quantile-quantile plots. In this
study, we focused on the most commonly used residual plot as a starting
point for exploring the new field of automated visual inference.
Similarly, we did not explore other, more sophisticated computer vision
model architectures and specifications for the same reason. While the
performance of the computer vision model is acceptable, there is still
room for improvement to achieve behavior more closely resembling that of
humans interpreting residual plots. This may require external survey
data or human subject experiment data to understand the fundamental
differences between our implementation and human evaluation.

\section{Conclusion}\label{conclusion}

In this study, we have introduced a distance measure based on
Kullback-Leibler divergence to quantify the disparity between the
residual distribution of a fitted classical normal linear regression
model and the reference residual distribution assumed under correct
model specification. This distance measure effectively captures the
magnitude of model violations in misspecified models. We propose a
computer vision model to estimate this distance, utilizing the residual
plot of the fitted model as input. The resulting estimated distance
serves as the foundation for constructing a single Model Violation Index
(MVI), facilitating the quantification of various model violations.

Moreover, the estimated distance enables the development of a formal
statistical testing procedure by evaluating a large number of null plots
generated from the fitted model. Additionally, employing bootstrapping
techniques and refitting the regression model allows us to ascertain how
frequently the fitted model is considered misspecified if data were
repeatedly obtained from the same data generating process.

The trained computer vision model demonstrates strong performance on
both the training and test sets, although it exhibits slightly lower
performance on residual plots with non-linearity visual patterns
compared to other types of violations. The statistical tests relying on
the estimated distance predicted by the computer vision model exhibit
lower sensitivity compared to conventional tests but higher sensitivity
compared to visual tests conducted by humans. While the estimated
distance generally mirrors the strength of the visual signal perceived
by humans, there remains scope for further improvement in its
performance.

Several examples are provided to showcase the effectiveness of the
proposed method across different scenarios, emphasizing the similarity
between visual tests and distance-based tests. Overall, both visual
tests and distance-based tests can be viewed as combined tests, aiming
to assess the correctness of all model assumptions collectively. In
contrast, individual residual diagnostic tests such as the RESET test
and the Breusch-Pagan test only evaluate specific model assumptions. In
practice, selecting an appropriate set of statistical tests for
regression diagnostics can be challenging, particularly given the
necessity of adjusting the significance level for each test.

Our method holds significant value as it helps alleviate a portion of
analysts' workload associated with assessing residual plots. While we
recommend analysts to continue reading residual plots whenever feasible,
as they offer invaluable insights, our approach serves as a valuable
tool for automating the diagnostic process or for supplementary purposes
when needed.

\section*{Acknowledgements}\label{acknowledgements}
\addcontentsline{toc}{section}{Acknowledgements}

These \texttt{R} packages were used for the work: \texttt{tidyverse}
\citep{tidyverse}, \texttt{lmtest} \citep{lmtest}, \texttt{mpoly}
\citep{mpoly}, \texttt{ggmosaic} \citep{ggmosaic}, \texttt{kableExtra}
\citep{kableextra}, \texttt{patchwork} \citep{patchwork},
\texttt{cassowaryr} \citep{mason2022cassowaryr}, \texttt{bandicoot}
\citep{bandicoot}, \texttt{ggpcp} \citep{ggpcp}, \texttt{keras}
\citep{keras} and \texttt{here} \citep{here}.

The article was created with R packages \texttt{rticles}
\citep{rticles}, \texttt{knitr} \citep{knitr} and \texttt{rmarkdown}
\citep{rmarkdown}. The project's GitHub repository
(\url{https://github.com/TengMCing/auto_residual_reading/paper})
contains all materials required to reproduce this article.

\section*{Disclosure Statement}\label{disclosure-statement}
\addcontentsline{toc}{section}{Disclosure Statement}

No potential conflict of interest was reported by the author(s).

\section*{Supplementary Materials}\label{supplementary-materials}
\addcontentsline{toc}{section}{Supplementary Materials}

\begin{description}
\item{Appendix:} The appendix includes more details about the neural network layers used for constructing the computer vision models.
\end{description}

\bibliographystyle{tfcad}
\bibliography{ref.bib}





\end{document}
