% interactcadsample.tex
% v1.03 - April 2017

\documentclass[]{interact}

\usepackage{epstopdf}% To incorporate .eps illustrations using PDFLaTeX, etc.
\usepackage{subfigure}% Support for small, `sub' figures and tables
%\usepackage[nolists,tablesfirst]{endfloat}% To `separate' figures and tables from text if required

\usepackage{natbib}% Citation support using natbib.sty
\bibpunct[, ]{(}{)}{;}{a}{}{,}% Citation support using natbib.sty
\renewcommand\bibfont{\fontsize{10}{12}\selectfont}% Bibliography support using natbib.sty

\theoremstyle{plain}% Theorem-like structures provided by amsthm.sty
\newtheorem{theorem}{Theorem}[section]
\newtheorem{lemma}[theorem]{Lemma}
\newtheorem{corollary}[theorem]{Corollary}
\newtheorem{proposition}[theorem]{Proposition}

\theoremstyle{definition}
\newtheorem{definition}[theorem]{Definition}
\newtheorem{example}[theorem]{Example}

\theoremstyle{remark}
\newtheorem{remark}{Remark}
\newtheorem{notation}{Notation}


% tightlist command for lists without linebreak
\providecommand{\tightlist}{%
  \setlength{\itemsep}{0pt}\setlength{\parskip}{0pt}}



\usepackage{lscape}
\usepackage{hyperref}
\usepackage[utf8]{inputenc}
\def\tightlist{}
\usepackage{setspace}


\begin{document}


\articletype{ARTICLE TEMPLATE}

\title{Automated reading of residual plots with computer vision models}


\author{\name{Weihao Li$^{a}$}
\affil{$^{a}$Department of Econometrics and Business Statistics, Monash
University, Clayton, VIC, Australia}
}

\thanks{CONTACT Weihao
Li. Email: \href{mailto:weihao.li@monash.edu}{\nolinkurl{weihao.li@monash.edu}}}

\maketitle

\begin{abstract}
TBD.
\end{abstract}

\begin{keywords}
TBD
\end{keywords}

\hypertarget{introduction}{%
\section{Introduction}\label{introduction}}

Residuals, within regression analysis, represent the differences between
fitted values and observed data points, capturing the unexplained
elements in the regression model. The practice of plotting residuals,
advocated by influential regression literature
\citep{cook1982residuals, draper1998applied, belsley1980regression, montgomery1982introduction},
serves as a standard procedure in regression diagnostics. This visual
examination is crucial for identifying deviations from the model
assumptions like linearity, homoscedasticity, and normality.

Generating a residual plot in most statistical software is often as
straightforward as executing a line of code or clicking a button.
However, accurately interpreting a residual plot can be challenging.
Consider Figure \ref{fig:false-finding} as an example, the residuals
display a triangular shape pointing to the left. While this might
suggest heteroskedasticity, it's important to avoid over-interpreting
the visual pattern. In this case, the fitted model is correctly
specified, and the triangular shape is actually a result of the skewed
distribution of the predictors, rather than indicating a flaw in the
model.

A residual plot can exhibit various visual features, but it's crucial to
recognize that some may arise from the characteristics of predictors and
the inherent randomness of the error, rather than indicating a violation
of model assumptions \citep{li2023plot}. The concept of visual
inference, as proposed by \citet{buja2009statistical}, provides an
inferential framework to assess whether residual plots indeed contain
visual patterns inconsistent with the model assumptions. The fundamental
idea involves testing whether the actual residual plot significantly
differs visually from null plots, which are created using residuals
generated from the null distribution. Typically, this is accomplished
through the lineup protocol. In this approach, the real residual plot is
embedded within a lineup alongside several null plots. If the real
residual plot can be distinguished from the lineup, it provides evidence
for rejecting the null hypothesis.

Delivering a residual plot as a lineup is generally considered a good
practice. However, as pointed out by \citet{li2023plot}, a primary
limitation of the lineup protocol is its reliance on human judgments.
Unlike conventional statistical tests that can be performed numerically
and automatically in statistical software, the lineup protocol requires
human evaluation of images. This characteristic makes it less suitable
for large-scale applications, given the associated high labour costs and
time requirements.

In this paper, we attempt to address this issue by integrating computer
vision models into residual plot diagnostics. The paper is structured as
follows: \ldots{}

\begin{figure}

{\centering \includegraphics[width=1\linewidth]{paper_files/figure-latex/false-finding-1} 

}

\caption{An example residual vs fitted values plot (red line indicates 0). The vertical spread of the data points varies with the fitted values. This often indicates the existence of heteroskedasticity.}\label{fig:false-finding}
\end{figure}

\hypertarget{methodogology}{%
\section{Methodogology}\label{methodogology}}

\hypertarget{generation-of-simulated-training-data}{%
\subsection{Generation of simulated training
data}\label{generation-of-simulated-training-data}}

\hypertarget{architecture-of-the-computer-vision-model}{%
\subsection{Architecture of the computer vision
model}\label{architecture-of-the-computer-vision-model}}

\hypertarget{training-process-and-hyperparameter-tuning}{%
\subsection{Training process and hyperparameter
tuning}\label{training-process-and-hyperparameter-tuning}}

\hypertarget{results}{%
\section{Results}\label{results}}

\hypertarget{model-evaluation}{%
\subsection{Model evaluation}\label{model-evaluation}}

\begin{itemize}
\tightlist
\item
  Metrics for model performance
\item
  Shap values
\item
  Heatmap
\end{itemize}

\hypertarget{comparison-with-human-visual-inference}{%
\subsection{Comparison with human visual
inference}\label{comparison-with-human-visual-inference}}

\hypertarget{overview-of-the-human-subject-experiment}{%
\subsubsection{Overview of the human subject
experiment}\label{overview-of-the-human-subject-experiment}}

\hypertarget{comparison}{%
\subsubsection{Comparison}\label{comparison}}

\begin{itemize}
\tightlist
\item
  power comparison
\item
  decisions
\end{itemize}

\hypertarget{case-study-1}{%
\subsection{Case study 1: \ldots{}}\label{case-study-1}}

\hypertarget{case-study-2}{%
\subsection{Case study 2: \ldots{}}\label{case-study-2}}

\hypertarget{case-study-3-datasaurus}{%
\subsection{Case study 3: datasaurus}\label{case-study-3-datasaurus}}

\hypertarget{dicussion}{%
\section{Dicussion}\label{dicussion}}

\hypertarget{conclusion}{%
\section{Conclusion}\label{conclusion}}

\begin{itemize}
\tightlist
\item
  Summary of findings
\item
  Contributions to the field
\item
  Future directions for research
\end{itemize}

\bibliographystyle{tfcad}
\bibliography{ref.bib}





\end{document}
